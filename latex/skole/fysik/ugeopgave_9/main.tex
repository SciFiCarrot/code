\documentclass[11pt,a4paper]{article}
\usepackage[danish]{babel}
\usepackage[T1]{fontenc}
\usepackage{enumitem}
\usepackage{array, tabularx}
\usepackage[table]{xcolor}
\usepackage{colortbl}
\usepackage{unicode-math}
\usepackage{pgfplots}
\usepackage{tikz}
\usepackage{xurl}            
\usepackage[hidelinks]{hyperref}
\usepackage{lipsum}
\usepackage[backend=biber,
            style=apa,
            sorting=nyt]{biblatex}
\addbibresource{kilder.bib}
\usepackage{csquotes}

\urlstyle{same}                 
\hypersetup{
  colorlinks=true,
  linkcolor=black,
  urlcolor=blue,                
  citecolor=black
}
\urlstyle{same}
\pgfplotsset{compat=1.18}

\setlist[enumerate,1]{
    label=\mbox{},        
    leftmargin=0pt,       
    labelsep=0pt,
    align=left,
    itemsep=1.5\baselineskip 
}
\setlist[enumerate,2]{
    label=(\alph*),
    leftmargin=2em,
    labelsep=.6em,
    itemsep=\baselineskip
}

\title{Ugeopgave 9}
\author{Luis Parker Noah Conradty}


\begin{document}
    \maketitle
    \begin{enumerate}
        \item
            \begin{enumerate}
                \item 
                    Der er to givende værdier: \(I = 6,11A\) og \(U= 0,366V\).
                    Ifølge Ohms lov gælder der:
                    \[
                        U = R· I ⇒ R = \frac UI = \frac{0,366V}{6,11A} = \boxed{59,9mΩ}
                    \]
                    Resistansen i amperemeteret er altså ca. \(60 mΩ\)
                \item
                    Stromstyrken forøges nu til til \(I = 100A\), amperemeteret har et indre resistans af \(R = 53mΩ\)
                    og viser \(I = 10A\). Der blev også tilføjet et Shuntmodstand \(R_S\).\\
                    Kredsløbet med Amperemederet og Shuntmodstanden er parallel, derfor er spændingen i de to gren lige. 
                    Spændingen i den nedere er:
                    \[U = R · I = 0,053Ω ·10A =\boxed{ 0,53V}\]
                    Strømstyrken fordeler sig over de to gren, og det gælder at: \(I_1 + I_2 = I_{alt} = 100A
                    ⇒I_1 = 100A - 10A = 90A\), 
                    Resistansen af Shuntmodstanden kan nu beregnes:
                    \[R_S = \frac{U}{I_1} = \frac{0,53V}{90A}=0,0059Ω = \boxed{5,9mΩ}\]
                \item
                    Den samlede effekt er: 
                    \begin{align*}
                        P_{sam} &= P_1 + P_2 \\
                        P_{1,2} &= U · I_{1,2} \\
                        P_{sam} &= (U · I_1) + (U · I_2)\\
                                &= U · (I_1 + I_2)\\
                                &= 0,53V · 100A \\
                                &= \boxed{53W}
                    \end{align*}




            \end{enumerate}
    \end{enumerate}

\end{document}

