\documentclass[11pt,a4paper]{article}
\usepackage[danish]{babel}
\usepackage[T1]{fontenc}
\usepackage{enumitem}
\usepackage{array, tabularx}
\usepackage[table]{xcolor}
\usepackage{colortbl}
\usepackage{unicode-math}
\usepackage{pgfplots}
\usepackage{tikz}
\usepackage{xurl}            
\usepackage[hidelinks]{hyperref}
\usepackage{lipsum}
\usepackage[backend=biber,
            style=apa,
            sorting=nyt]{biblatex}
\addbibresource{kilder.bib}
\usepackage{csquotes}
\usepackage{circuitikz}

\urlstyle{same}                 
\hypersetup{
  colorlinks=true,
  linkcolor=black,
  urlcolor=blue,                
  citecolor=black
}
\urlstyle{same}
\pgfplotsset{compat=1.18}

\setlist[enumerate,1]{
    label=(\alph*),
    leftmargin=2em,
    labelsep=.6em,
    itemsep=\baselineskip
}

\title{Ugeopgave 8}
\author{Luis Parker Noah Conradty}


\begin{document}
    \maketitle
    \begin{enumerate}
        \item
            Det her er kredsløbet. Man kan måle spændingen over Led'en og over resistoren,
            og stromstyrken over det hele.

            \begin{figure}[h]
                \begin{center}
                    \begin{circuitikz}
                        \draw  
                            (0,4) to [battery1] (0,0)
                                to [generic, l_=$330Ω$] (6,0) 
                            (0,4) to [led] (6,4)
                                to [ammeter] (6,0)
                            (2,4) to [short] (2,3)
                                to [voltmeter] (4,3)
                                to [short] (4,4)
                            (2,0) to [short] (2,1)
                                to [voltmeter] (4,1)
                               to [short] (4,0);
                    \end{circuitikz}
                \end{center}
            \end{figure}

        \item
            Formellen til forholden mellem $I$, $U$ og $R$ er:
            \[U=R·I ⇒ R = \frac{U} I = \frac{6,2V}{330Ω} \approx 0,0188A \]
            Efter indsætning af værdierne kommer man frem til $0,0188A$.
        \item
            Strømstyrken over hele kredsløbet er det samme  \(\left(\boxed{0,0188A} \right)\) fordi alt er i serie.
            Det betyder at det er den samme strømstyrke gennem resistoren og igennem Led'en er i serie.
            Spændingen kan beregnes på en anden måde, her tager vi hele spændingen og trækker den fra resitorens spænding.
            Dvs. \(U_{led}=U_{alt}-U_{R} = 9V-6,2V=\boxed{2,8V}\) er spændingen over led'en. 
            Effekten kan beregnes ved at gange de to værdier med hinanden:
            \[P = I · U_{led} = 0,0188A · 2,8V = \boxed{0,053W}\]
        \item
            Det er en rød Led, fordi dens 'Forward Voltage' er ca. \(1,8V-2V\) og det svarer til en rød Led
            (\href{https://www.valiantekworld.com/Electronics_and_Robotics/Led_Circuit.php}{Kilde, Figur 2}).

    \end{enumerate}
\end{document}


