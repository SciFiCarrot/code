\documentclass[11pt,a4paper]{article}
\usepackage[danish]{babel}
\usepackage{enumitem}
\usepackage{array, tabularx}
\usepackage[table]{xcolor}
\usepackage{colortbl}
\usepackage{unicode-math}
\usepackage{pgfplots}
\usepackage{tikz}
\usepackage{xurl}            
\usepackage[hidelinks]{hyperref}   
\urlstyle{same}
\pgfplotsset{compat=1.18}

\setlist[enumerate,1]{
    label=\mbox{},        % empty (invisible) label
    leftmargin=0pt,       % align exactly with the text
    labelsep=0pt,
    align=left,
    itemsep=1.5\baselineskip % vertical gap between Opgaver
}
\setlist[enumerate,2]{
    label=(\alph*),
    leftmargin=2em,
    labelsep=.6em,
    itemsep=\baselineskip
}

\title{Fysik Aflevering 4}
\author{Luis Conradty}


\begin{document}
    \maketitle
    \begin{enumerate}
        \item{Opgave 2}
            \begin{enumerate}
                \item
                    Fotonens energi beregnes ved formlen:
                    \[E = \frac{hc}{λ} 
                        = \frac{6,626 ·10^{-34}·Js · 2,99 \dfrac{m}{s}· 10^8}{589·10^{-9}m}
                        = 3,36 · 10^{-19} 
                    \]
                \item 
                    Den kritiske vinkel kan beregnes ved formelen:
                    \[
                        \sin(I_c) = \frac{n_2}{n_1}⇒ n_2 = {n_1}·{\sin(I_c)}
                    \]
                    Da vi leder efter \(n_2\) her, kan vi bare indsette vores værdierne:
                    \[ n_2 = 1,7681 · \sin(51,474^\circ) = 1,3832\]
                    Brydningsfaktor af væsken er altså 1,3832

            \end{enumerate}}
    \end{enumerate}
\end{document}

