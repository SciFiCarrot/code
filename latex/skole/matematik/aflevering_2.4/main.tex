\documentclass[11pt,a4paper]{article}
\usepackage[danish]{babel}
\usepackage{amsfonts}
%\usepackage[T1]{fontenc}
\usepackage{enumitem}
\usepackage{array, tabularx}
\usepackage[table]{xcolor}
\usepackage{colortbl}
\usepackage{unicode-math}
\usepackage{pgfplots}
\usepackage{tikz}
\usepackage{xurl}            
\usepackage[hidelinks]{hyperref}
\usepackage{lipsum}
\usepackage[backend=biber,
            style=apa,
            sorting=nyt]{biblatex}
\addbibresource{kilder.bib}
\usepackage{csquotes}

\urlstyle{same}                 
\hypersetup{
  colorlinks=true,
  linkcolor=black,
  urlcolor=blue,                
  citecolor=black
}
\urlstyle{same}
\pgfplotsset{compat=1.18}

\setlist[enumerate,1]{
    label=\arabic*.,        
    leftmargin=0pt,       
    labelsep=0pt,
    align=left,
    itemsep=1.5\baselineskip 
}
\setlist[enumerate,2]{
    label=(\alph*),
    leftmargin=2em,
    labelsep=.6em,
    itemsep=\baselineskip
}

\title{Matematik Aflevering 2.4}
\author{Luis Parker Noah Conradty \& Carl Viggo Riesenhuber Hofman}


\begin{document}
    \maketitle
    \begin{enumerate}
        \item
            Den givende rekursionsligningen er:
            \[
                x_{n+2} - 6x_{n+1} + 9x_n = 0    
            \]
            \begin{enumerate}
                \item
                    Den kararistiske ligning er:
                    \[r^{2} - 6r + 9r = 0\]
                    den løses som en andengradsligning:
                    \[r_{1,2} = \frac{6 \pm \sqrt{36 - 4 · 9}}{2} = 3 \pm 0 = 3 \]
                    Det er en reel rod, og den fuldstændige løsning er givet efter Sætining 3.2.6, som:
                    \[
                        x_n =\boxed{C·3^n + D·n·3^n}, \quad C,D \in \mathbb{R}
                    \]
                \item
                    For at finde den partikulere løsing med \(x_0 = 1, \ x_1 = 2\), skal vi løse de to ligninger:
                    \[
                        \begin{aligned}
                            x_0 = 1:\quad 1 &= C\cdot 3^{0} + D\cdot 0\cdot 3^{0} = C
                            &&\Rightarrow\quad C=1,\\[4pt]
                            x_1 = 2:\quad 2 &= C\cdot 3^{1} + D\cdot 1\cdot 3^{1} = 3C+3D
                            &&\Rightarrow\quad 2 = 3 + 3D
                            \Rightarrow D=-\frac13.
                        \end{aligned}
                    \]

                    \begin{align*}
                        x_n &= 3^n - \frac{1}3 ·n · 3^n \\
                        x_n &= 3^n - \frac{n·3^n}3
                    \end{align*}

                        

            \end{enumerate}
    \end{enumerate}
    
\end{document}

