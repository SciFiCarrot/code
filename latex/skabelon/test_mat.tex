\documentclass[11pt,a4paper]{article}

% --- Sprog & layout ---
\usepackage[danish]{babel}
\usepackage[T1]{fontenc}
\usepackage[a4paper,margin=2.2cm]{geometry}
\usepackage{microtype}
\usepackage[hidelinks]{hyperref}

% --- Matematik ---
\usepackage{amsmath,amssymb,mathtools}

% --- Pæn struktur ---
\usepackage{enumitem}
\setlist{nosep}

\usepackage{fancyhdr}
\pagestyle{fancy}
\fancyhf{}
\lhead{\textbf{Matematik A} — Talfølger 1}
\rhead{\textbf{Aflevering}}
\cfoot{\thepage}

% --- Udfyld selv ---
\newcommand{\Dato}{5.\ december 2025}
\newcommand{\NavnA}{Navn 1}
\newcommand{\NavnB}{Navn 2 (valgfri)}
\newcommand{\Klasse}{Klasse}
\newcommand{\Lærer}{Lærer}
\newcommand{\Titel}{Aflevering: Talfølger 1}

% --- Små “CAS-note” bokse (valgfrit) ---
\usepackage[most]{tcolorbox}
\newtcolorbox{casnote}{
  colback=white, colframe=black, boxrule=0.6pt, arc=2mm,
  title=CAS-note (forklar hvad CAS gør for dig)
}

\begin{document}

\begin{center}
  {\LARGE \textbf{\Titel}}\\[2mm]
  {\large \Dato}\\[3mm]
\end{center}

\noindent
\textbf{Navn(e):} \NavnA \ifx\NavnB\empty\else\ , \NavnB\fi \\
\textbf{Klasse:} \Klasse \qquad
\textbf{Lærer:} \Lærer

\vspace{4mm}
\hrule
\vspace{6mm}

% =========================
% OPGAVE 1
% =========================
\section*{Opgave 1}
Vi betragter rekursionsligningen
\[
x_{n+2} - 6x_{n+1} + 9x_n = 0.
\]

\subsection*{(a) Fuldstændig løsning}
\textbf{Idé/metode:} (fx karakteristisk ligning)\\
\textbf{Udregning:}
\[
\text{(skriv din udledning her)}
\]
\textbf{Konklusion:} (skriv den generelle løsning tydeligt)

\subsection*{(b) Partikulær løsning med $x_0=1$ og $x_1=2$}
\textbf{Indsæt startbetingelser:}\\
\[
\text{(vis hvordan du finder konstanterne)}
\]
\textbf{Svar:} (skriv $x_n=$ ...)

% =========================
% OPGAVE 2
% =========================
\section*{Opgave 2}
Du indsætter $100000$ kr på en opsparingskonto med årlig rente på $2\%$.
Lad $x_n$ være beløbet på kontoen $n$ år efter indsættelsen.

\subsection*{(a) Rekursionsligning og startbetingelse}
\textbf{Forklaring:} (hvorfor netop denne rekursion?)\\
\[
x_{n+1} = \dots
\]
\[
x_0 = \dots
\]

\subsection*{(b) Partikulær løsning}
\textbf{Udledning:}\\
\[
x_n = \dots
\]

\subsection*{(c) Saldo efter 10 år}
\[
x_{10} = \dots
\]
\textbf{Svar:} \dots\ kr

\subsection*{(d) Hvornår når saldoen 150000 kr?}
\textbf{Ulighed:}\\
\[
x_n \ge 150000
\]
\textbf{Løsning for $n$:}\\
\[
n \ge \dots
\]
\textbf{Svar:} \dots\ år (afrunding/argumentation)

\begin{casnote}
Hvis du bruger CAS her: skriv præcis hvad du bad CAS om (fx løsning af ulighed),
og hvordan du fortolker resultatet i konteksten (heltal, afrunding, osv.).
\end{casnote}

% =========================
% OPGAVE 3
% =========================
\section*{Opgave 3}
Bier og forfædre (han/hun). Lad $b_n$ være antallet af bier i den $n$'te generation før en given bi.

\subsection*{(a) Stamtræ 4 generationer baglæns + $b_0,\dots,b_4$}
\textbf{Tegning/argumentation:}\\
(Sæt evt. en figur ind eller lav en tydelig tekstbeskrivelse.)\\[1mm]
\textbf{Værdier:}
\[
b_0=\dots,\quad b_1=\dots,\quad b_2=\dots,\quad b_3=\dots,\quad b_4=\dots
\]

\subsection*{(b) Vis/forklar hvorfor $y_{n+1}=x_n$}
\textbf{Forklaring:} \\
\[
\text{(skriv argumentationen her)}
\]

\subsection*{(c) Forklar hvorfor $x_{n+1}=b_n=x_n+y_n$}
\textbf{Forklaring:}\\
\[
\text{(skriv argumentationen her)}
\]

\subsection*{(d) Vis at $b_{n+2}=b_{n+1}+b_n$ og find fuldstændig løsning}
\textbf{Udledning fra (b) og (c):}\\
\[
\text{(trin-for-trin)}
\]
\textbf{Fuldstændig løsning:}\\
\[
b_n=\dots
\]

\subsection*{(e) Startbetingelser for hunbi: forklar $b_1=2$ og $b_2=3$}
\textbf{Forklaring:}\\
\[
\text{(skriv argumentationen her)}
\]

\subsection*{(f) Startbetingelser for 4 hanbier og 3 hunbier + $b_{10}$}
\textbf{Startdata:}\\
\[
x_0=3,\quad y_0=4 \quad \Rightarrow \quad b_0=x_0+y_0=\dots
\]
\[
b_1=\dots,\quad b_2=\dots \quad \text{(angiv passende startbetingelser)}
\]
\textbf{Beregning af $b_{10}$:}\\
\[
b_{10}=\dots
\]
\textbf{Svar:} \dots\ forfædre i 10. generation før dem.

\vspace{6mm}
\hrule
\vspace{3mm}

\section*{Kilder / Sætninger brugt}
Skriv hvilke sætninger/definitioner du bruger (fx om løsning af lineære rekursioner, renteformler osv.).

\end{document}
