\documentclass[11pt,a4paper]{article}

% ---------- Language & Fonts ----------
\usepackage[english]{babel} % change to [danish] if you want
\usepackage[T1]{fontenc}
\usepackage[utf8]{inputenc} % harmless even if modern engines ignore it
\usepackage{lmodern}

% ---------- Page layout ----------
\usepackage[a4paper,margin=2.2cm]{geometry}
\usepackage{setspace}
\setstretch{1.08}

% ---------- Math ----------
\usepackage{amsmath,amssymb,amsthm,mathtools}

% ---------- Figures / Tables ----------
\usepackage{graphicx}
\usepackage{float}
\usepackage{booktabs}
\usepackage{tabularx}

% ---------- Lists ----------
\usepackage{enumitem}
\setlist{nosep}

% ---------- Links ----------
\usepackage[hidelinks]{hyperref}

% ---------- Code ----------
\usepackage{listings}
\usepackage{xcolor}
\lstset{
  basicstyle=\ttfamily\small,
  numbers=left,
  numberstyle=\tiny,
  stepnumber=1,
  breaklines=true,
  frame=single,
  tabsize=2
}

% ---------- Nice boxes for problems/solutions ----------
\usepackage[most]{tcolorbox}
\tcbset{
  colback=white,
  colframe=black,
  boxrule=0.8pt,
  arc=2mm,
  left=2mm,right=2mm,top=1mm,bottom=1mm
}

% ---------- Header / Footer ----------
\usepackage{fancyhdr}
\pagestyle{fancy}
\fancyhf{}
\lhead{\textbf{\CourseCode} — \CourseName}
\rhead{\textbf{\AssignmentTitle}}
\cfoot{\thepage}

% ---------- User info (edit these) ----------
\newcommand{\CourseCode}{COURSE123}
\newcommand{\CourseName}{Course Name}
\newcommand{\AssignmentTitle}{Assignment \#1}
\newcommand{\StudentName}{Your Name}
\newcommand{\StudentID}{Student ID}
\newcommand{\Teacher}{Teacher / Examiner}
\newcommand{\DueDate}{2025-12-12}

% ---------- Environments ----------
\newcounter{problem}
\newenvironment{problem}[1][]{
  \refstepcounter{problem}
  \begin{tcolorbox}[title={Problem \theproblem\if\relax\detokenize{#1}\relax\else\ ( #1 )\fi}]
}{
  \end{tcolorbox}
}

\newenvironment{solution}{
  \begin{tcolorbox}[title={Solution}]
}{
  \end{tcolorbox}
}

\newcommand{\points}[1]{\hfill\textbf{[#1 pts]}}

% ---------- Document ----------
\begin{document}

% Title block
\begin{tcolorbox}
  {\LARGE \textbf{\AssignmentTitle}}\\[2mm]
  \textbf{Course:} \CourseCode\ — \CourseName \\
  \textbf{Student:} \StudentName \quad (\StudentID) \\
  \textbf{Teacher:} \Teacher \\
  \textbf{Due date:} \DueDate
\end{tcolorbox}

% Optional: short intro
\section*{Notes (optional)}
Write assumptions, given data, or what you are going to do.

% ---------- Problems ----------
\begin{problem}[Short title]\points{10}
State the task (you can paste the question here).
\end{problem}

\begin{solution}
Your answer goes here.

\textbf{Example math:}
\[
  \sum_{k=1}^{n} k = \frac{n(n+1)}{2}.
\]

\textbf{Example list:}
\begin{enumerate}
  \item Step one
  \item Step two
\end{enumerate}

\textbf{Example figure:}
\begin{figure}[H]
  \centering
  \includegraphics[width=0.7\linewidth]{example-image}
  \caption{Caption here.}
\end{figure}

\textbf{Example table:}
\begin{table}[H]
  \centering
  \begin{tabular}{lrr}
    \toprule
    Item & Value A & Value B \\
    \midrule
    Foo  & 1       & 2 \\
    Bar  & 3       & 4 \\
    \bottomrule
  \end{tabular}
  \caption{A simple table.}
\end{table}

\textbf{Example code:}
\begin{lstlisting}[language=Python, caption={Example code}]
def f(n):
    return n*(n+1)//2
\end{lstlisting}

\end{solution}

% Copy-paste this block for more problems:
% \begin{problem}[...]\points{...}
% ...
% \end{problem}
% \begin{solution}
% ...
% \end{solution}

\end{document}
